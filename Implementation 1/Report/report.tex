%%%%%%%%%%%%%%%%%%%%%%%%%%%%%%%%%%%%%%%%%
% Minimalist Book Title Page 
% LaTeX Template
% Version 1.0 (27/12/12)
%
% This template has been downloaded from:
% http://www.LaTeXTemplates.com
%
% Original author:
% Peter Wilson (herries.press@earthlink.net)
%
% License:
% CC BY-NC-SA 3.0 (http://creativecommons.org/licenses/by-nc-sa/3.0/)
% 
% Instructions for using this template:
% This title page compiles as is. If you wish to include this title page in 
% another document, you will need to copy everything before 
% \begin{document} into the preamble of your document. The title page is
% then included using \titleTH within your document.
%
%%%%%%%%%%%%%%%%%%%%%%%%%%%%%%%%%%%%%%%%%

%----------------------------------------------------------------------------------------
%	PACKAGES AND OTHER DOCUMENT CONFIGURATIONS
%----------------------------------------------------------------------------------------

%\title{Microprocessor Architecture : Labo 1}

\documentclass{article}

\usepackage[utf8]{inputenc}
\usepackage[T1]{fontenc}
\usepackage[svgnames]{xcolor} % Required to specify font color
\usepackage{mathpazo}
\usepackage{floatrow}
\usepackage{geometry}%réglages mise en page
\geometry{%
a4paper, % note : l'option a4paper tuait la marge supérieure.
body={170mm,250mm}, %
left=25mm,top=25mm,right=25mm, %
headheight=21mm,headsep=7mm,
marginparsep=4mm,
marginparwidth=20mm, %
footnotesep=50mm
}
\usepackage{longtable}
\usepackage{pdflscape}
% allows for temporary adjustment of side margins
\usepackage{chngpage}
\usepackage{graphicx}
\usepackage{float}
\usepackage{color}
\usepackage{amssymb}

\definecolor{pblue}{rgb}{0.13,0.13,1}
\definecolor{pgreen}{rgb}{0,0.5,0}
\definecolor{pred}{rgb}{0.9,0,0}
\definecolor{pgrey}{rgb}{0.46,0.45,0.48}

\usepackage{listings}
\lstset{ % 
  language=R,                % the language of the code 
  basicstyle=\footnotesize,           % the size of the fonts that are used for the code 
  numbers=left,                   % where to put the line-numbers 
  numberstyle=\tiny\color{gray},  % the style that is used for the line-numbers 
  stepnumber=2,                   % the step between two line-numbers. If it's 1, each line 
                                  % will be numbered 
  numbersep=5pt,                  % how far the line-numbers are from the code 
  backgroundcolor=\color{white},      % choose the background color. You must add \usepackage{color} 
  showspaces=false,               % show spaces adding particular underscores 
  showstringspaces=false,         % underline spaces within strings 
  showtabs=false,                 % show tabs within strings adding particular underscores 
  frame=single,                   % adds a frame around the code 
  rulecolor=\color{black},        % if not set, the frame-color may be changed on line-breaks within not-black text (e.g. commens (green here)) 
  tabsize=2,                      % sets default tabsize to 2 spaces 
  captionpos=b,                   % sets the caption-position to bottom 
  breaklines=true,                % sets automatic line breaking 
  breakatwhitespace=false,        % sets if automatic breaks should only happen at whitespace 
  %title=\lstname,                   % show the filename of files included with \lstinputlisting; 
                                  % also try caption instead of title 
  keywordstyle=\color{blue},          % keyword style 
  commentstyle=\color{dkgreen},       % comment style 
  stringstyle=\color{purple},         % string literal style 
  escapeinside={\%*}{*)},            % if you want to add a comment within your code 
  morekeywords={*,...}               % if you want to add more keywords to the set 
} 


\newcommand*{\course}{\fbox{INFO-H-413}} % Generic publisher logo

%----------------------------------------------------------------------------------------
%	TITLE PAGE
%----------------------------------------------------------------------------------------

\newcommand*{\titleTH}{\begingroup % Create the command for including the title page in the document
\raggedleft % Right-align all text
\vspace*{\baselineskip} % Whitespace at the top of the page

{\Large \textsc{Anthony Debruyn}}\\[0.167\textheight] % Author name

{\LARGE\bfseries Heuristic Optimisation}\\[\baselineskip] % First part of the title, if it is unimportant consider making the font size smaller to accentuate the main title

{\textcolor{Orange}{\Huge Implementation Exercise 1}}\\[\baselineskip] % Main title which draws the focus of the reader

{\Large \textit{Iterative improvement algorithms for the PFSP}}\par % Tagline or further description

\vfill % Whitespace between the title block and the publisher

%\vspace*{30\baselineskip} % Whitespace at the bottom of the page

{\large Dr. F. Mascia, Dr. T. Stützle \course}\par % Publisher and logo

%\vspace*{5\baselineskip} % Whitespace at the bottom of the page
\endgroup}

%----------------------------------------------------------------------------------------
%	BLANK DOCUMENT
%----------------------------------------------------------------------------------------

\begin{document} 

\thispagestyle{empty}

\titleTH % This command includes the title page

\newpage

\section{Introduction}
The goal of this assignment was to...

\section{Code Use}


\section{Exercise 1.1}

\subsection{The Results}
We list here the average percentage deviation from the best solutions for each algorithm tested, along with the average computation time:

\begin{figure}[H]
\begin{center}
	\begin{tabular}{|c|c|c|} \hline
		Algorithm & APD & ACT(ms) \\ \hline \hline
		
		--random--exchange--best		& 26		& 1362 \\ \hline
		--random--exchange--first	& 10		& 5744 \\ \hline
		--random--insert--best		& 16		& 2909 \\ \hline
		--random--insert--first		& 6		& 12869 \\ \hline
		--random--transpose--best	& 354	& 28 \\ \hline
		--random--transpose--first	& 340	& 37 \\ \hline
		--slack--exchange--best		& 32		& 1244 \\ \hline
		--slack--exchange--first		& 11		& 3944 \\ \hline
		--slack--insert--best		& 18		& 2523 \\ \hline
		--slack--insert--first		& 6		& 8661 \\ \hline
		--slack--transpose--best		& 243	& 43 \\ \hline
		--slack--transpose--first	& 240	& 47 \\ \hline
	\end{tabular}
\end{center}
\caption{The APD and ACT for each algorithm.}
\label{al}
\end{figure}

We see that the transpose mode for the neighbourhood generation is the worst, and by far, but is the less costly in time. The insert mode is the best when it comes to the quality of the solution, but is the worst in time cost. The remaining exchange mode is between the other ones, half less time than the insert mode, and a quality of solution a bit lower than the insert mode, but very far from the quality of the transpose mode.\\

The reason behind the difference between these 3 modes becomes clear when we consider the size of the respective neighbourhood. If N is the number of jobs and p is the current permutation, we get these sizes:
$$|N_{transpose}(p)| = N-1$$
$$|N_{exchange}(p)| = \sum_{i=0}^{N-2}{N-i-1} = (N-1)^2 - \frac{(N-1)(N-2)}{2}$$
$$|N_{insert}(p)| = N(N-1)$$

We observe that the SLACK heuristic makes the time cost decrease everywhere, except for the transpose mode. As an attempt to explain this, we could say that the size of the neighbourhood, being a lot smaller than the other mode (linear in the number of jobs), has a much lower impact on the time cost. The real impact comes from the heuristic at the beginning.\\

The APD is always lower for the "first" mode than the "best" mode. Would this be explained by the fact that more diverse solutions are found with the first mode ? Indeed, we see that the time consumed for the first mode is much superior than the time used for the best mode. And if we look at the number of intermediate solutions, the number is much higher. We could say that with the best mode, we explore more locally than with the first mode, and get stuck quickly in a local maximum.

\subsection{Difference between the solutions}
We used the Student t-test to determine whether there there is a statistically significant difference between the solutions generated by the different perturbative local search algorithms. The obtained p-values for the Student t-test are shown in figure \ref{pv st}. The algorithms corresponding to the numbers are the ones from figure \ref{al}, in the same order (alphabetically sorted).\\

The Student t-test can be used to test statistically the mean equality null hypothesis. The p-value corresponds to the probability that the null hypothesis is incorrectly rejected. If this value is below the significance level, the null hypothesis is rejected and thus the means are very different. If the value is above the significance level, the hypothesis is incorrectly rejected, thus accepted, and the means are approximately equal. The higher the value, the closer the means.\\

Here is the R code used to write the p-values to a file. The results matrix was created by taking the average percentage deviation of each algorithm and each instance (matrix of dimension 60x12).

\begin{figure}[H]
\begin{lstlisting}
a <- read.table("R-avRelPer--random--exchange--best.dat")$V1
...
l <- read.table("R-avRelPer--slack--transpose--first.dat")$V1

results <- c(a,b,c,d,e,f,g,h,i,j,k,l)
results <- array(results, dim=c(60,12))

test = numeric(length = 60)
test2 = numeric(length = 60)

x = 1
for (i in 1:11) {
	for (j in (i+1):12) {
		test[x] <- t.test (results[,i], results[,j], paired=T)$p.value
		test2[x] <- wilcox.test (results[,i], results[,j], paired=T)$p.value
		x = x + 1
	}
}

write(test, file = "p values st test", ncolumns = 1)
write(test2, file = "wilcox p-values", ncolumns = 1)
\end{lstlisting}
\end{figure}

We take a significance level of 0.05 ($\alpha = 0.05$).


\begin{landscape}
\begin{figure}
\begin{adjustwidth}{-1cm}{-1cm}
\begin{center}
	\begin{tabular}{|c|c|c|c|c|c|c|c|c|c|c|c|c|} \hline
		& 1 & 2 & 3 & 4 & 5 & 6 & 7 & 8 & 9 & 10 & 11 & 12 \\ \hline \hline
		1 & & 2.325677e-10 & 0.0001764137 & 1.492921e-10 & 2.324708e-07 & 1.397426e-08 & 0.009320388 & 1.814296e-10 & 0.0009812694 & 1.651045e-12 & 1.002514e-09 & 3.956704e-09 \\ \hline
		2 & &  & 1.380619e-07 & 5.676628e-05 & 1.667657e-07 & 1.022813e-08 & 1.119052e-10 & 0.2236329 & 1.656816e-12 & 1.904083e-08 & 6.981541e-10 & 2.61601e-09 \\ \hline
		3 & & &  & 2.616446e-12 & 2.784541e-07 & 1.86176e-08 & 4.364759e-06 & 6.281086e-05 & 0.2084699 & 9.391734e-12 & 2.116133e-09 & 7.112663e-09 \\ \hline
		4 & & & &  & 1.560699e-07 & 1.017382e-08 & 5.164549e-10 & 4.937766e-05 & 6.505518e-17 & 0.3771135 & 7.162175e-10 & 2.629253e-09 \\ \hline
		5 & & & & &  & 0.3131815 & 3.198631e-07 & 1.624651e-07 & 2.896557e-07 & 1.378438e-07 & 0.0002024202 & 5.743684e-05 \\ \hline
		6 & & & & & &  & 1.81788e-08 & 1.005804e-08 & 2.054526e-08 & 8.50651e-09 & 1.793673e-05 & 7.765756e-07 \\ \hline
		7 & & & & & & &  & 4.646304e-11 & 3.193926e-05 & 1.330835e-11 & 1.541866e-09 & 5.902744e-09 \\ \hline
		8 & & & & & & & & & 6.202973e-11 & 1.221381e-07 & 6.366285e-10 & 2.46713e-09 \\ \hline
		9 & & & & & & & & & & 8.756858e-19 & 1.987018e-09 & 7.336251e-09 \\ \hline
		10 & & & & & & & & & & & 5.066557e-10 & 1.945128e-09 \\ \hline
		11 & & & & & & & & & & & & 0.4696791 \\ \hline
		12 & & & & & & & & & & & & \\ \hline

	
	\end{tabular}
\end{center}
\end{adjustwidth}
\caption{P-values for each combination of algorithms (Student t-test). Paired test, with each possible pair of algorithms.}
\label{pv st}
\end{figure}

\begin{figure}
\begin{adjustwidth}{-1cm}{-1cm}
\begin{center}
	\begin{tabular}{|c|c|c|c|c|c|c|c|c|c|c|c|c|} \hline
		& 1 & 2 & 3 & 4 & 5 & 6 & 7 & 8 & 9 & 10 & 11 & 12 \\ \hline \hline
		1 & & 1.6479e-11 & 6.566262e-07 & 1.650747e-11 & 1.66871e-11 & 1.668972e-11 & 0.001887301 & 3.558256e-11 & 3.880961e-05 & 1.649194e-11 & 1.66871e-11 & 1.66688e-11 \\ \hline
		2 & &  & 1.401733e-07 & 5.791151e-06 & 1.670281e-11 & 1.670805e-11 & 1.646866e-11 & 0.7562894 & 7.25813e-09 & 1.681026e-08 & 1.668972e-11 & 1.667926e-11 \\ \hline
		3 & & &  & 1.218116e-10 & 1.669234e-11 & 1.670281e-11 & 2.685764e-09 & 9.410355e-06 & 0.07339366 & 1.376776e-09 & 1.669757e-11 & 1.667664e-11 \\ \hline
		4 & & & &  & 1.669757e-11 & 1.667664e-11 & 1.648417e-11 & 4.229889e-07 & 2.965186e-11 & 0.861429 & 1.670281e-11 & 1.668449e-11 \\ \hline
		5 & & & & &  & 0.7742204 & 1.669757e-11 & 1.669495e-11 & 1.670019e-11 & 1.670543e-11 & 4.520244e-11 & 3.454208e-11 \\ \hline
		6 & & & & & &  & 1.66871e-11 & 1.669495e-11 & 1.669495e-11 & 1.669757e-11 & 1.66688e-11 & 2.449249e-11 \\ \hline
		7 & & & & & & &  & 1.642735e-11 & 5.420053e-09 & 1.657235e-11 & 1.670281e-11 & 1.669757e-11 \\ \hline
		8 & & & & & & & & & 2.515346e-08 & 3.401544e-08 & 1.670019e-11 & 1.670019e-11 \\ \hline
		9 & & & & & & & & &  & 5.411351e-11 & 1.66871e-11 & 1.667403e-11 \\ \hline
		10 & & & & & & & & & & & 1.670019e-11 & 1.669495e-11 \\ \hline
		11 & & & & & & & & & & & & 0.0002718787 \\ \hline
		12 & & & & & & & & & & & & \\ \hline

	
	\end{tabular}
\end{center}
\end{adjustwidth}
\caption{P-values for each combination of algorithms (Wilcoxon test). Paired test, with each possible pair of algorithms.}
\label{pv wilcox}
\end{figure}
\end{landscape}

The Wilcoxon test does not assume that the population is normally distributed. We see on figure \ref{pv wilcox} that the p-values give the same idea of the difference between the solutions, except for the combination 11x12. We can see that the hypothesis is rejected in the Wilcoxon test but not in the Student t-test.\\

We can see on the figures that a lot of values are below the threshold $\alpha$. We can thus conclude that most of the algorithms generate solutions of significantly different quality. We can check it in figure \ref{al}.


\section{Exercise 1.2}

\end{document}