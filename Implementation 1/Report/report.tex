%%%%%%%%%%%%%%%%%%%%%%%%%%%%%%%%%%%%%%%%%
% Minimalist Book Title Page 
% LaTeX Template
% Version 1.0 (27/12/12)
%
% This template has been downloaded from:
% http://www.LaTeXTemplates.com
%
% Original author:
% Peter Wilson (herries.press@earthlink.net)
%
% License:
% CC BY-NC-SA 3.0 (http://creativecommons.org/licenses/by-nc-sa/3.0/)
% 
% Instructions for using this template:
% This title page compiles as is. If you wish to include this title page in 
% another document, you will need to copy everything before 
% \begin{document} into the preamble of your document. The title page is
% then included using \titleTH within your document.
%
%%%%%%%%%%%%%%%%%%%%%%%%%%%%%%%%%%%%%%%%%

%----------------------------------------------------------------------------------------
%	PACKAGES AND OTHER DOCUMENT CONFIGURATIONS
%----------------------------------------------------------------------------------------

%\title{Microprocessor Architecture : Labo 1}

\documentclass{article}

\usepackage[utf8]{inputenc}
\usepackage[T1]{fontenc}
\usepackage[svgnames]{xcolor} % Required to specify font color
\usepackage{mathpazo}
\usepackage{floatrow}
\usepackage{geometry}%réglages mise en page
\geometry{%
a4paper, % note : l'option a4paper tuait la marge supérieure.
body={170mm,250mm}, %
left=25mm,top=25mm,right=25mm, %
headheight=21mm,headsep=7mm,
marginparsep=4mm,
marginparwidth=20mm, %
footnotesep=50mm
}
\usepackage{longtable}
\usepackage{pdflscape}
% allows for temporary adjustment of side margins
\usepackage{chngpage}
\usepackage{graphicx}
\usepackage{float}
\usepackage{color}
\usepackage{amssymb}


\newcommand*{\course}{\fbox{INFO-H-413}} % Generic publisher logo

%----------------------------------------------------------------------------------------
%	TITLE PAGE
%----------------------------------------------------------------------------------------

\newcommand*{\titleTH}{\begingroup % Create the command for including the title page in the document
\raggedleft % Right-align all text
\vspace*{\baselineskip} % Whitespace at the top of the page

{\Large \textsc{Anthony Debruyn}}\\[0.167\textheight] % Author name

{\LARGE\bfseries Heuristic Optimisation}\\[\baselineskip] % First part of the title, if it is unimportant consider making the font size smaller to accentuate the main title

{\textcolor{Orange}{\Huge Implementation Exercise 1}}\\[\baselineskip] % Main title which draws the focus of the reader

{\Large \textit{Iterative improvement algorithms for the PFSP}}\par % Tagline or further description

\vfill % Whitespace between the title block and the publisher

%\vspace*{30\baselineskip} % Whitespace at the bottom of the page

{\large Dr. F. Mascia, Dr. T. Stützle \course}\par % Publisher and logo

%\vspace*{5\baselineskip} % Whitespace at the bottom of the page
\endgroup}

%----------------------------------------------------------------------------------------
%	BLANK DOCUMENT
%----------------------------------------------------------------------------------------

\begin{document} 

\thispagestyle{empty}

\titleTH % This command includes the title page

\newpage

\section{Introduction}

\section{Code Use}

\section{Exercise 1.1}

\subsection{The Results}
We list here the average percentage deviation from the best solutions for each algorithm tested, along with the average computation time:

\begin{center}
	\begin{tabular}{|c|c|c|} \hline
		Algorithm & APD & ACT(ms) \\ \hline \hline
		
		--random--exchange--best		& 289	& 111 \\ \hline
		--random--exchange--first	& 295	& 38 \\ \hline
		--random--insert--best		& 313	& 100 \\ \hline
		--random--insert--first		& 277	& 133 \\ \hline
		--random--transpose--best	& 411	& 5 \\ \hline
		--random--transpose--first	& 416	& 12 \\ \hline
		--slack--exchange--best		& 241	& 110 \\ \hline
		--slack--exchange--first		& 255	& 34 \\ \hline
		--slack--insert--best		& 223	& 103 \\ \hline
		--slack--insert--first		& 200	& 120 \\ \hline
		--slack--transpose--best		& 299	& 16 \\ \hline
		--slack--transpose--first	& 308	& 16 \\ \hline
	\end{tabular}
\end{center}

\subsection{Difference between the solutions}
We used the Student t-test to determine whether there there is a statistically significant difference between the solutions generated by the different perturbative local search algorithms.

\section{Exercise 1.2}

\end{document}